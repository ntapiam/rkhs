\documentclass[10pt, reqno]{article}
\usepackage{amsthm}
\usepackage[sfdefault]{FiraSans}
\usepackage[utf8]{inputenc}
\usepackage[T1]{fontenc}
\usepackage{newtxsf}
\usepackage{bookmark}
\usepackage[ocgcolorlinks]{ocgx2}
\usepackage{mathtools,mleftright,subdepth}
\usepackage[nameinlink]{cleveref}
\usepackage{cite}
\usepackage[scr=stixtwofancy,bb=stixtwo]{mathalfa}
\usepackage[margin={3cm,2cm}]{geometry}

\theoremstyle{definition}
\newtheorem{definition}{Definition}
\newtheorem{example}{Example}

\theoremstyle{plain}
\newtheorem{theorem}{Theorem}
\newtheorem{proposition}{Proposition}


\newcommand{\R}{\mathbb{R}}
\newcommand{\T}{\mathscr{T}}
\newcommand{\N}{\mathbb{N}}
\newcommand{\bbX}{\mathbb{X}}
\newcommand{\X}{\mathscr{X}}
\renewcommand{\H}{\mathscr{H}}
\newcommand{\fX}{\mathfrak{X}}
\DeclarePairedDelimiter{\dpar}{(\mkern-4mu(}{)\mkern-3.5mu)}

\mleftright
\begin{document}
\section{Reproducing Kernel Hilbert Spaces}
We briefly recall some definitions. Throughout this section \(\X\) denotes an arbitrary set
\begin{definition}
	A symmetric function \(k\colon\X\times\X\to\R\) is called a \emph{positive-definite} kernel on \(\X\) if
	\[
		\sum_{i=1}^n\sum_{j=1}^nc_ic_jk(x_i,x_j)\ge 0
	\]
	holds for any given \(n\in\N\), \(c_1,\dotsc,c_n\in\R\) and \(x_1,\dotsc,x_n\in\X\).
\end{definition}
\begin{definition}
	Let \(\H\) be a Hilbert space of real-valued functions on \(\X\).
	We say that \(\H\) is a reproducing kernel Hilbert space (RKHS) if, for all \(x\in\X\) the evaluation functional \(\H\ni f\mapsto f(x)\) is
	continuous.
\end{definition}

By Riesz's theorem, for every \(x\in\X\) there is a unique element \(K_x\in\H\) such that \(f(x)=\langle f, K_x\rangle\) for all \(f\in\H\).
Since \(K_x\) is itself an element of \(\H\), we see that \(K_x(y)=\langle K_x, K_y\rangle\).
This allows us to define the \emph{reproducing kernel} of \(\H\) by \(k\colon\X\times\X\to\R\) as
\[
	k(x,y)\coloneqq \langle K_x,K_y\rangle.
\]
One can show that \(k\) must be symmetric and positive definite.

\begin{theorem}[Moore--Aronszajn]
	\label{thm:MA}
	Suppose that \(k\) is a symmetric and positive definite kernel on \(\X\). There is a unique Hilbert space \(\H\) of functions on \(\X\) for which \(k\)
	is a reproducing kernel.
\end{theorem}

The space \(\H\) can be expcitly constructed: denote \(K_x\coloneq k(x,\cdot)\) and let \(\H^\circ\) be the linear span of \(\{K_x:x\in\X\}\). Define an inner product on \(\H^\circ\) by
\[
	\langle K_x,K_y\rangle=k(x,y).
\]
Finally, let \(\H\) be the completion of \(\H^\circ\) with respect to this inner product.


\begin{example}
	Consider \(\X=[0,1]\) and the kernel \(k(s,t)=s\wedge t-st\).
	This kernel is clearly symmetric.
\end{example}

\subsection{Tensor Hilbert space}
Let \((\R^d)^{\otimes k}\) denote the \(k\)-fold tensor product of \(\R^d\) with itself, and denote by
\[
	\mathsf{T}\dpar{\R^d}\coloneqq\prod_{n=0}^\infty( \R^d )^{\otimes k}
\]
the (complete) tensor algebra over \(\R^d\), i.e., the space of formal tensor series.
Denote
\[
	\T\coloneqq\left\{ a\in\mathsf{T}\dpar{\R^d}:\sum_{n=1}^{\infty}\|a_n\|^2_{(\R^d)^{\otimes k}}<+\infty\right\}.
\]
For each \(k\), \((\R^d)^{\otimes k}\) becomes a Hilbert space when endowed with the ``natural''\footnote{In the sense that this is the natural way of
tensoring arbitrary Hilbert spaces.} inner product
\[
	\langle e_{i_1\dotsm i_n},e_{j_1\dotsm j_m}\rangle\coloneqq\delta_{n=m}\prod_{k=1}^n\langle e_{i_k},e_{j_k}\rangle
\]
where \((e_{i}:i=1,\dotsc,d)\) is an ONB of \(\R^d\) and \(e_{i_1\dotsm i_n}\) stands for the tensor product \(e_{i_1}\otimes\dotsm\otimes
e_{i_n}\).
Then, \(\T\) can be viewed as the direct sum of these individual Hilbert spaces, and becomes itself a Hilbert space with the product
\[
	\langle a,b\rangle\coloneqq\sum_{n=1}\langle a_n,b_n\rangle_{(\R^d)^{\otimes n}}.
\]

\subsection{The signature kernel}
Fix \(L\in (0,1)\) and consider the space \(\X\) to be the ball of radius \(L\) in \(BV\).
Given \(X\in\X\), denote \(\bar{X}_t\coloneqq(X_t,\tfrac{1-L}{2}t)\).
It can be shown that taking signatures of elements of \(\X\) yields elements in \(\T\), in the sense that for all
\(X\in\X\) we have that \(S(\bar X)\in\T\).

Now, define on \(\X\) the following kernel:
\[
	k(X,Y)\coloneqq\langle S(\bar X),S(\bar Y)\rangle_{\T}.
\]
Then, \Cref{thm:MA} yields the following RKHS:
\[
	\H=\overline{\mathrm{span}\left\{ \xi_\alpha\colon\X\to\R\mid\xi_\alpha(X)=\langle \alpha,S(\bar X)\rangle,\alpha\in\T \right\}}
\]
with inner product \(\langle\xi_\alpha,\xi_\beta\rangle_{\N}=\langle \alpha,\beta\rangle_{\T}.\)

\subsection{Relation to CDEs/ResNets}
\subsubsection{Composition of vector fields}
Let \(\fX=\fX(\R^n)\) denote the space of smooth vector fields \(\R^n\).
Define the bilinear operator \(\triangleleft\colon\fX\times\fX\to\fX\)
by
\[
	(f\triangleleft g)(x)=Dg(x)f(x),\quad x\in\R^n.
\]
We note that this operation is not associative, but it satisfies the left pre-Lie relation:
considering the associator
\(a_\triangleleft(f,g,h)\coloneqq f\triangleleft(g\triangleleft h)-(f\triangleleft g)\triangleleft h\),
we have that it is symmetric in \(f\) and \(g\).
Indeed
\begin{align*}
	a_\triangleleft(f,g,h)&= D^2h(x)(f(x),g(x)) + Dh(x)Dg(x)f(x) - Dh(x)Dg(x)f(x)\\
	&= D^2h(x)(f(x),g(x)).
\end{align*}
We define \(r^{(n)}_\triangleleft(f_1,\dotsc,f_n)\coloneqq f_1\triangleleft r^{(n-1)}_\triangleleft(f_2,\dotsc,f_n)\).

Since we will be dealing with compositions of vector fields and the chain rule, we essentially need to understand the
\emph{universal enveloping algebra} \(\mathsf{U}(\fX)\) of \(\fX\). This is the natural space to deal with polynomials in elements of
\(\fX\) that ``respect the (pre-)Lie structure''.
Luckily, Guin and Oudom already described this object for pre-Lie algebras.

Recall that any pre-Lie algebra gives rise to a Lie algebra via anti-symmetrization of the pre-Lie product:
\[
	[f,g]\coloneqq f\triangleleft g.
\]
In our example this is exactly the Jacobi bracket of vector fields.
By definition, \(\mathsf{U}(\fX)\) is the associative algebra obtained by taking the quotient of the
tensor algebra \(\mathsf{T}(\fX)\) by the two-sided ideal generated by the relation \([f,g]=f\otimes g-g\otimes f\).

Consider the following extension of \(\triangleleft\) to \(S(\fX)\times S(\fX)\): for \(A,B\in S(\fX)\)
\begin{align*}
	1\triangleleft A&= A,\quad A\triangleleft 1=0\\
	fA\triangleleft g&= f\triangleleft (A\triangleleft g)-(f\triangleleft A)\triangleleft g\\
	C\triangleleft AB&= (C_{(1)}\triangleleft A)(C_{(2)}\triangleleft B).
\end{align*}
where \(\Delta C=C_{(1)}\otimes C_{(2)}\) in \(S(\fX)\) (unshuffle coproduct).
\begin{proposition}
	The following identities hold:
	\begin{itemize}
		\item Let \(f,g_1,\dotsc,g_n\in\fX\), then:
			\[
				f\triangleleft(g_1\dotsm g_n)=\sum_{i=1}^ng_1\dotsm(f\triangleleft g_i)\dotsm g_n.
			\]
		\item Let \(f_1,\dotsc,f_n,g\in\fX\), then:
			\[
				(f_1\dotsm f_n)\triangleleft g=D^ng(f_1,\dotsc,f_n).
			\]
	\end{itemize}
\end{proposition}

We now introduce a non-commutative (!) product on \(S(\fX)\) given by
\[
	f\star g=f_{(1)}(f_{(2)}\triangleleft g).
\]
It can be shown that this product is associative.
\begin{proposition}
	Let \(f_1,\dotsc,f_n,g\in\fX\). Then
	\[
		r^{(n+1)}_{\triangleleft}(f_1,\dotsc,f_n,g)=(f_1\star\dotsm\star f_n)\triangleleft g.
	\]
\end{proposition}

\begin{example}
	Let \(f_1,f_2,g\in\fX\). Then
	\begin{align*}
		r^{(3)}_\triangleleft(f_1,f_2,g)&= f_1\triangleleft(f_2\triangleleft g)\\
		&= f_1\triangleleft Dg\cdot f_2\\
		&= D^2g(f_1,f_2)+Dg\cdot Df_2\cdot f_1\\
		&= (f_1f_2)\triangleleft g+(f_1\triangleleft f_2)\triangleleft g\\
		&= (f_1f_2+f_1\triangleleft f_2)\triangleleft g\\
		&= (f_1\star f_2)\triangleleft g
	\end{align*}
	where the last equality follows from the fact that \(f_1\in\fX\) is primitive in \(S(\fX)\).
\end{example}
\begin{proof}[Proof of Proposition 2]
	By induction:
	\begin{align*}
		r^{(n+1)}_\triangleleft(f_1,\dotsc,f_n,g)&= f_1\triangleleft r^{(n)}_\triangleleft(f_2,\dotsc,f_n,g)\\
		&= f_1\triangleleft ((f_2\star\dotsm\star f_n)\triangleleft g).
	\end{align*}
	Now, using the definition of \(\triangleleft\) on \(S(\fX)\) we deduce that
	\[
		r^{(n+1)}_\triangleleft(f_1,\dotsc,f_n,g)=(f_1\triangleleft(f_2\star\dotsm\star f_n))\triangleleft g + (f_1(f_2\star\dotsm\star f_n))\triangleleft g.
	\]
	The proof is finished by using the definition of the \(\star\) product and the fact that \(f_1\in\fX\) is primitive in
	\(S(\fX)\).
\end{proof}

\begin{theorem}
	The universal enveloping algebra \(\mathsf{U}(\fX)\) is isomorphic, as a Hopf algebra, to \((S(\fX),\star,\Delta)\).
\end{theorem}

\subsection{Solving RDEs}
Let \(z\) solve \(\dot z=f(z)\,\mathrm dX\).
We know that for any small enough \(T>0\) and smooth observable \(\varphi\colon\R^n\to\R^n\), the value of \(\varphi(z_T)\) can be
well approximated by a Taylor expansion of the form
\[
	\varphi(z_T)\approx\sum_{k=0}^N\sum_{i_1,\dotsc,i_k=1}^dr^{(k+1)}_\triangleleft(f_{i_1},\dotsc,f_{i_k},\varphi)(z_0)S(\bar X)^{i_1\dotsm i_k}_{0,T}.
\]
This means that we can get a good approximation by a \emph{linear} functional of the signature, i.e., an element
\(\xi_\alpha\in\H\), defined by the element \(\alpha\in\T\) given by
\[
	\langle \alpha,i_1\dotsm i_k\rangle=r^{(k+1)}_\triangleleft(f_{i_1},\dotsc,f_{i_k},\varphi)(z_0)=(f_{i_1}\star\dotsm\star f_{i_k})\triangleleft\varphi(z_0),
\]
so that
\[
	\varphi(z_T)\approx\xi_\alpha(X).
\]
\end{document}

% vim:spell:spelllang=en_us:tw=120
