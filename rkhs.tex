% arara: pdflatex
% arara: bibtex
\documentclass[10pt, reqno]{article}
\usepackage{amsthm}
\usepackage[sfdefault]{FiraSans}
\usepackage[utf8]{inputenc}
\usepackage[T1]{fontenc}
\usepackage{newtxsf}
\usepackage{bookmark}
\usepackage[ocgcolorlinks]{ocgx2}
\usepackage{mathtools}
\usepackage[nameinlink]{cleveref}
\usepackage{cite}
\usepackage[scr=stixtwofancy,bb=stixtwo]{mathalfa}
\usepackage[margin={3cm,2cm}]{geometry}

\theoremstyle{definition}
\newtheorem{definition}{Definition}

\theoremstyle{plain}
\newtheorem{theorem}{Theorem}


\newcommand{\R}{\mathbb{R}}
\newcommand{\T}{\mathsf{T}}

\begin{document}
\section{Reproducing Kernel Hilber Spaces}
We briefly recall the definition.
\begin{definition}
	Let \(X\) be an arbitrary set, and \(H\) a Hilbert space of real-valued functions on \(X\).
	We say that \(H\) is a reproducing kernel Hilbert space (RKHS) if, for all \(x\in X\) the evaluation functional \(H\ni f\mapsto f(x)\) is
	continuous.
\end{definition}

By Riesz's theorem, for every \(x\in X\) there is a unique element \(K_x\in H\) such that \(f(x)=\langle f, K_x\rangle\) for all \(f\in H\).
Since \(K_x\) is itself an element of \(H\), we see that \(K_x(y)=\langle K_x, K_y\rangle\).
This allows us to define the \emph{reproducing kernel} of \(H\) by \(k\colon X\times X\to\R\) as
\[
	k(x,y)\coloneqq \langle K_x,K_y\rangle.
\]
One can show that \(k\) must be symmetric and positive definite.

\begin{theorem}[Moore--Aronszajn]
	\label{thm:MA}
	Suppose that \(k\) is a symmetric and positive definite kernel on \(X\). There is a unique Hilbert space \(H\) of functions on \(X\) for which \(k\)
	is a reproducing kernel.
\end{theorem}

The space \(H\) can be expcitly constructed: denote \(K_x\coloneq k(x,\cdot)\) and let \(H^\circ\) be the linear span of \(\{K_x:x\in X\}\). Define an inner product on \(H^\circ\) by
\[
	\langle K_x,K_y\rangle=k(x,y).
\]
Finally, let \(H\) be the completion of \(H^\circ\) with respect to this inner product.

\subsection{Tensor Hilbert space}
Let \((\R^d)^{\otimes k}\) denote the \(k\)-fold tensor product of \(\R^d\) with itself, and denote by
\[
	\T(\R^d)\coloneqq\prod_{n=0}^\infty( \R^d )^{\otimes k}
\]
the (complete) tensor algebra over \(\R^d\), i.e., the space of formal tensor series.
Denote
\[
	\mathscr{T}\coloneqq\left\{ a\in\T:\sum_{n=1}^{\infty}\|a_n\|^2_{(\R^d)^{\otimes k}}<+\infty\right\}.
\]
For each \(k\), \((\R^d)^{\otimes k}\) becomes a Hilbert space when endowed with the ``natural''\footnote{In the sense that this is the natural way of
tensoring arbitrary Hilbert spaces.} inner product
\[
	\langle e_{i_1\dotsm i_n},e_{j_1\dotsm j_m}\rangle\coloneqq\delta_{n=m}\prod_{k=1}^n\langle e_{i_k},e_{j_k}\rangle
\]
where \((e_{i}:i=1,\dotsc,d)\) is an ONB of \(\R^d\) and \(e_{i_1\dotsm i_n}\) stands for the tensor product \(e_{i_1}\otimes\dotsm\otimes
e_{i_n}\).
Then, \(\mathscr{T}\) can be viewed as the direct sum of these individual Hilbert spaces, and becomes itself a Hilbert space with the product
\[
	\langle a,b\rangle\coloneqq\sum_{n=1}\langle a_n,b_n\rangle_{(\R^d)^{\otimes n}}.
\]

\subsection{The signature kernel}
Fix \(L\in (0,1)\) and consider the space \(\mathscr{X}\) to be the ball of radius \(L\) in \(BV\).
Given \(X\in\mathscr{X}\), denote \(\bar{X}_t\coloneqq(X_t,\tfrac{1-L}{2}t)\).
It can be shown that taking signatures of elements of \(\mathscr{X}\) yields elements in \(\mathscr{T}\), in the sense that for all
\(X\in\mathscr{X}\) we have that \(S(\bar X)\in\mathscr{T}\).

Now, define on \(\mathscr{X}\) the following kernel:
\[
	k(X,Y)\coloneqq\langle S(\bar X),S(\bar Y)\rangle_{\mathscr{T}}.
\]
Then, \Cref{thm:MA} yields the following RKHS:
\[
	\mathscr{H}=\overline{\left\{ \xi_\alpha\colon\mathscr{X}\to\R\mid\xi_\alpha(X)=\langle \alpha,S(\bar X)\rangle,\alpha\in\mathscr{T} \right\}}
\]
with inner product \(\langle\xi_\alpha,\xi_\beta\rangle_{\mathscr{H}}=\langle \alpha,\beta\rangle_{\mathscr{T}}.\)

\subsection{Relation to CDEs/ResNets}
Let \(z\) solve \(\dot z=f(z)\,\mathrm dX\).
We know that for any \(T>0\), the value of \(z_T\) is well approximated by a linear functional of the signature of \(\bar{X}\), i.e., an element of
\(\mathscr{H}\), to wit: \( z_T\approx\xi_\alpha(\bar{X}) \) where \(\alpha\in\mathscr{T}\) satisfies
\[
	\langle\alpha, e_{i_1\dotsm i_n}\rangle = \frac{1}{n!}f_{i_1}\circ\dotsm\circ f_{i_n}(z_0).
\]
\end{document}
